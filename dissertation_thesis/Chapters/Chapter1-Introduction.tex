\chapter{Introduction}\label{chap_intro} % Main chapter title

\label{Chapter1} % For referencing this chapter elsewhere, use \ref{Chapter1}

\lhead{Chapter 1. \emph{Introduction}}

%----------------------------------------------------------------------------------------
%	SECTION 1.1 - Context
%----------------------------------------------------------------------------------------

\section{Context}

Ever since the creation of computers, their computation power has been a source of progress in terms of high-precision scientific computations in one hand or efficient energy expensive mass computations. In one case or the other, this performance has been met through the development of more and more performant hardware architectures. Moore forecasted the evolution through his law announced in 1965. However, the last decade has proven to counterbalance this law and makes it meet an end due to physical and thermal limitations. This decrease in hardware evolution has rather shifted the focus to other vectors of performance. One of them is parallelisation through the use of multiple or a cluster of computers. The goal is then to increase the computational power. This focus has enabled the development of new architectures tailored to those specific tasks, such as Graphical Processing Units (GPUs) or reprogrammable architectures such as Field Programmable Gate Arrays (FPGAs). In parallel, the use of reduced precision to increase the performance of computational-expensive tasks is a focus of the literature that can be applied particularly well to state-of-the-art hardware architectures. Next chapters look over the ideas under mixed-precision and the ways to implement them efficiently. Next derive applications that can benefit from the implementations and their translation to physical architectures.

%-----------------------------------
%	SECTION 1.2 - Aims and Objectives
%-----------------------------------
\section{Aims and Objectives}

The aim of the thesis is to give the reader an insight of the mixed-precision mindset and intentions by covering important articles and ideas the literature holds. The practical aim and result of the literature review is to implement a specific application using mixed-precision. This application consists of deploying a \emph{quantised neural network} (i.e. a neural network using reduced-precision) on a \emph{reconfigurable architecture}.

The resulting aims are the following. The objectives will be defined after the literature review in the \guille{Requirements analysis} chapter.

\textbf{Aims}
\begin{enumerate}
  \item Provide the reader with notions in \emph{Number Representation}.
  \item Provide the reader with notions in \emph{Machine Learning}.
  \item Provide the reader with notions in \emph{Hardware Architecture}.
  \item Link the three pools of notions through the \emph{Literature Review}.
  \item Present the objective of the project: Deploy a \emph{neural network} on a \emph{reconfigurable architecture}.
  \item Present the mixed-precision \emph{motivations} and \emph{implementation methods}.
  \item Present the mixed-precision \emph{applications} to machine learning.
  \item Present the \emph{frameworks} and \emph{deployment solutions} for machine learning on reconfigurable architectures.
  \item Present an \emph{implementation of a benchmark} over a framework.
  \item Display \emph{results of the benchmarking}.
  \item Provide the reader with \emph{critics} on the results.
  \item Give out possible \emph{future works} or applications.
\end{enumerate}

%-----------------------------------
%	SECTION 1.3 - Structure of the report
%-----------------------------------

\section{Structure of the report}

The given thesis is structured as follows. \textbf{Chapter 1} provides a brief \emph{Introduction} to the context and objectives of the report. \textbf{Chapter 2} consists of the \emph{Background}, presenting the main ideas behind both \emph{number representation}, \emph{machine learning} and \emph{hardware architectures}. \textbf{Chapter 3} covers the \emph{Literature Review} and how \emph{mixed-precision} was implemented and used, how it can be applied to \emph{neural networks} and how \emph{neural networks} can be deployed on specific architectures. \textbf{Chapter 4} consists of a \emph{Requirement Analysis} of the development project of the dissertation. \textbf{Chapter 5} presents any \emph{Professional, Legal, Ethical and Social Issues} the project can highlight. \textbf{Chapter 6} consists of a presentation of the implementation of the project, it goes over the main points and presents the tools and frameworks used as well as the experiments performed. \textbf{Chapter 7} presents the results of the experiments while \textbf{Chapter 8} discusses the results and leads to future works that can extend the project. Finally, \textbf{Chapter 9} is the \emph{Conclusion} of the thesis.
