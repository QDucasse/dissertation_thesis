\chapter{Project Plan}

\label{Chapter5} For referencing this chapter elsewhere, use \ref{Chapter5}

\lhead{Chapter 5. \emph{Project Plan}}

%----------------------------------------------------------------------------------------
%	SECTION 1
%----------------------------------------------------------------------------------------

\section{Project Schedule}

The schedule of the project will consist of a list of tasks along with their estimated time. The best representation in order to output a graphical representation of those tasks and their deadline is to use a \emph{Gantt Chart}. This chart enables a graphical representation of the plan along with the allocated resources for each one of the tasks. This will allow to identify the critical path of the project - the sequence of tasks from beginning to end that takes the longest time to complete, any delay on one of the tasks of the sequence will automatically delay the whole project.

The tasks that will need to be completed by the end of the dissertation are the following:

\begin{itemize}
  \item Appropriation of the FPGA hardware architecture
  \begin{itemize}
    \item Basic actions
    \item Complete cycle
    \item Higher-level implementation
  \end{itemize}
  \item Appropriation of the CNN
  \begin{itemize}
    \item Basic implementation
    \item Parameter tuning
    \item Optimisation implementation
  \end{itemize}
  \item Combination of the CNN and FPGA
  \item Backend software
  \begin{itemize}
    \item Coordination between the two elements (hardware and network)
    \item Security rules
  \end{itemize}
  \item Experimentation guidelines
  \item Experimentation results
  \item Results Evaluation
  \item Dissertation writing
  \begin{itemize}
    \item Abstract
    \item Chapter 1 - Introduction
    \item Chapter 5 - Implementation
    \item Chapter 6 - Results
    \item Chapter 7 - Discussion
    \item Chapter 8 - Evaluation and Future Work
    \item Chapter 9 - Conclusion
  \end{itemize}
\end{itemize}

A Gantt chart has been created in order to comply with the fact that the final draft will have to be done by Friday 17/07. Backtracking from there, the other tasks and their associated resources can be viewed in \emph{Figure} \ref{fig:gantt}.

\begin{figure}[htbp]
	\centering
		\includegraphics[width=\textwidth]{Figures/gantt.png}
	\caption[Gantt Chart]{Gantt chart for the project}
	\label{fig:gantt}
\end{figure}

%----------------------------------------------------------------------------------------
%	SECTION 2
%----------------------------------------------------------------------------------------

\section{Supporting Plans and Risk Management}


The following project will be developed during an internship and will enable a loop of direct feedback from what I produce to the supervisors. The internship will help me produce a quality project as I will have to work by the side of my supervisors. If the subject will need me to learn new technologies and to implement a state-of-the-art application with them, I believe the environment will help me do so.

Any risk can be handle quickly due to the proximity with the supervisors. The most likely risk is the lack of completion of a requirement. This risk can occur due to several reasons: not detailed enough or too ambitious. In any case, the requirement will have to be rewritten and this modification will have to be logged and explained. Supervisors will have to be warned and they will have to confirm any change in the requirements.

The communication with supervisors will be simplified due to the special situation of the internship. Any issue can be handled directly in a closed-loop circuit. The situation will be unexpected due to special circumstances, the internship will begin remotely. This situation has to be handled correctly through meetings and weekly checks.
